%%%%%%%%%%%%%%%%%%%%%%%%%%%%%%%%%%%%%%%%%
% ---------------------------------------
% Beamer Presentation
% LaTeX Template
% Version 1.0 (10/11/12)
%
% This template has been downloaded from:
% http://www.LaTeXTemplates.com
% ---------------------------------------
% For FIT CTU presentatnion created by
% VLADISLAV BOBKO, 5th January, 2019
%
% License:
% CC BY-NC-SA 3.0 (     /)
%
%%%%%%%%%%%%%%%%%%%%%%%%%%%%%%%%%%%%%%%%%

\documentclass{beamer}

\mode<presentation>
{
        \usetheme{Madrid}
}

\usepackage{graphicx} % Allows including images
\usepackage{booktabs} % Allows the use of \toprule, \midrule and \bottomrule in                                                                                                                                tables

%-------------------------------------------------------------------------------                                                                                                                               ---------
%       TITLE PAGE
%-------------------------------------------------------------------------------                                                                                                                               ---------
\title[Webdav]{WebDav \& davfs2 }

\author{Vladislav Bobko}
\institute[FIT CVUT]
{
        Ceske Vysoke Uceni Technicke \\
        \medskip
        \textit{bobkovla@fitcvut.cz}
}
\date{January 5, 2019}

\begin{document}


\begin{frame}
\titlepage
\end{frame}

\begin{frame}
\frametitle{Overview}
\tableofcontents
\end{frame}

%-------------------------------------------------------------------------------                                                                                                                               ---------
%       PRESENTATION SLIDES
%-------------------------------------------------------------------------------                                                                                                                               ---------

%-------------------------------------------------------------------------------                                                                                                                               -----------------
\section{WEBDAV}
%-------------------------------------------------------------------------------                                                                                                                               -----------------

\subsection{Zakladni Informace} %
%-------------------------------------------------------------------------------                                                                                                                               -----------------
% PRVNI SLIDE - OBECNE INFO O WEBDAVU
\begin{frame}
\frametitle{ZAKLADNI INFORMACE }
    \begin{itemize}
        \item WEBDAV = Web Distributed Authoring and Versioning
        \item rozsireni HTTP protokolu
        \item Motivace - napr. zakaz FTP, proxy, autentifikace ...
        \item Verzovani - mozne pomoci Delta-V extension
    \end{itemize}
\end{frame}
%-------------------------------------------------------------------------------                                                                                                                               -----------------

% DRUHY SLIDE - MOZNE OPERACE
\subsection{Mozne operace}
%-------------------------------------------------------------------------------                                                                                                                               -----------------
\begin{frame}
    \frametitle{MOZNE OPERACE}
Krome zakladnich prikazu obsahuje navic:
    \begin{itemize}
        \item COPY
        \item MKCOL (= mkdir )
        \item MOVE
        \item PROPFIND
        \item PROPPATCH
        \item LOCK
        \item UNLOCK
    \end{itemize}
\end{frame}
%-------------------------------------------------------------------------------                                                                                                                               -----------------

% TRETI SLIDE - MOZNE OPERACE
\subsection{Stavove kody}
%-------------------------------------------------------------------------------                                                                                                                               -----------------
\begin{frame}
    \frametitle{STAVOVE KODY}
WebDav zavadi nektere nove stavy:
    \begin{itemize}
        \item 207 Multi-Status
        \item 422 Unprocessable Entity
        \item 423 Locked
        \item 424 Failed Dependency
        \item 507 Insufficient Storage
    \end{itemize}
\end{frame}
%-------------------------------------------------------------------------------                                                                                                                               -----------------
% CTVRTY SLIDE - DOTAZ KLIENTA - 1
\subsection{Priklad - dotaz klienta}
%-------------------------------------------------------------------------------                                                                                                                               -----------------
\begin{frame}[fragile] % Need to use the fragile option when verbatim is used in                                                                                                                                the slide
\frametitle{DOTAZ KLIENTA}
\begin{example}[Dotaz klienta - 1]
\begin{verbatim}
PROPPATCH /WebDavDocs/webdav-xml.htm HTTP/1.1
Host: myserver.com
Content-Type: text/xml
Content-Length: 138
   \end{verbatim}
\end{example}
\end{frame}
%-------------------------------------------------------------------------------                                                                                                                               -----------------

% PATY SLIDE - DOTAZ KLIENTA - 2
%-------------------------------------------------------------------------------                                                                                                                               -----------------
\begin{frame}[fragile]
\frametitle{DOTAZ KLIENTA}
\begin{example}[Dotaz klienta - 2]
\begin{verbatim}
<?xml version="1.0">
<d:propertyupdate xmlns:d="DAV:" xmlns:o="urn:schemas-
microsoft-com:office:office">
   <d:set>
      <d:prop>
         <o:Author>Sean Purcell</o:Author>
      </d:prop>
   </d:set>
</d:propertyupdate>
   \end{verbatim}
\end{example}
\end{frame}
%-------------------------------------------------------------------------------                                                                                                                               -----------------
% SESTY SLIDE - ODPOVED SERVERU - 1
\subsection{Priklad - odpoved serveru}
%-------------------------------------------------------------------------------                                                                                                                               -----------------
\begin{frame}[fragile]
\frametitle{ODPOVED SERVERU}
\begin{example}[Odpoved serveru - 1]
\begin{verbatim}
HTTP/1.1 207 Multi-Status
Server: Microsoft-IIS/5.0
Date: Wed, 04 Aug 1999 21:52:58 GMT
Content-Type: text/xml
Content-Length: 310
   \end{verbatim}
\end{example}
\end{frame}
%-------------------------------------------------------------------------------                                                                                                                               -----------------
% SEDMY SLIDE - ODPOVED SERVERU - 2
%-------------------------------------------------------------------------------                                                                                                                               -----------------
\begin{frame}[fragile]
\frametitle{ODPOVED SERVERU}
\begin{example}[Odpoved serveru - 2]
\begin{verbatim}
<?xml version="1.0"?>
<a:multistatus xmlns:b="urn:schemas-microsoft-com:office"
xmlns:a="DAV:">
  <a:response>
  <a:href>http://myserver.com/webdav/webdav-xml.htm</a:href>
      <a:propstat>
         <a:status>HTTP/1.1 200 OK</a:status>
         <a:prop>
            <b:Author/>
         </a:prop>
      </a:propstat>
   </a:response>
</a:multistatus>
   \end{verbatim}
\end{example}
\end{frame}
%-------------------------------------------------------------------------------                                                                                                                               -----------------
\section{davfs2}
%-------------------------------------------------------------------------------                                                                                                                               -----------------
% OSMY SLIDE - OVERVIEW DAFS2
%-------------------------------------------------------------------------------                                                                                                                               -----------------
\begin{frame}
\frametitle{davfs2}
\begin{itemize}
    \item{Linuxovy nastroj pro mount WebDav zdroju}
    \item Open-Source
    \item{https://savannah.nongnu.org/projects/davfs2}
\end{itemize}
\end{frame}
%-------------------------------------------------------------------------------                                                                                                                               -----------------

% DEVATY SLIDE - POUZITE ZDROJE
%-------------------------------------------------------------------------------                                                                                                                               -----------------
\begin{frame}
\frametitle{Zdroje}
    \begin{itemize}
        \item https://en.wikipedia.org/wiki/WebDAV
        \item https://wiki.archlinux.org/index.php/Davfs2
        \item http://savannah.nongnu.org/projects/davfs2
        \item http://www.webdav.org/
    \end{itemize}
\end{frame}

%------------------------------------------------

\begin{frame}
\Huge{\centerline{Konec}}
\end{frame}

%-------------------------------------------------------------------------------                                                                                                                               ---------

\end{document}